\documentclass[11pt]{article}
\usepackage[utf8]{inputenc}
\usepackage[T2A]{fontenc}
\usepackage[russian]{babel}
\usepackage{geometry}
\usepackage{tabularx}
\usepackage{booktabs}
\usepackage{enumitem}
\usepackage{titlesec}
\usepackage{caption}

% Настройка полей
\geometry{a4paper, margin=0.8in}

% Настройка заголовков
\titleformat{\section}{\large\bfseries}{\thesection}{1em}{}
\titleformat{\subsection}{\normalsize\bfseries}{\thesubsection}{1em}{}

% Убираем абзацный отступ для списков
\setlist[enumerate]{leftmargin=*, nosep, label=\arabic*.}

\begin{document}

{\centering\LARGE\bfseries Формирование бюджета разработки\par}
\vspace{1.5em}

\section{Оценка рисков}

\subsection{Матрица рисков}

\noindent
\begin{tabularx}{\textwidth}{lXXX}
\toprule
\textbf{Риск} & \textbf{Вероятность} & \textbf{Влияние} & \textbf{Уровень} \\
\midrule
\textbf{R1: Нестабильность внешних API} (Gutenberg, Google Books) & Средняя & Высокое & \textbf{Критический} \\
\textbf{R2: Сложности с реализацией офлайн-режима на Android} & Высокая & Среднее & \textbf{Высокий} \\
\textbf{R3: Задержки в разработке из-за болезней/отсутствия участников} & Низкая & Высокое & \textbf{Средний} \\
\textbf{R4: Низкий уровень удержания пользователей (Retention)} & Средняя & Высокое & \textbf{Высокий} \\
\textbf{R5: Конфликты в архитектуре при интеграции FE и BE} & Низкая & Среднее & \textbf{Низкий} \\
\bottomrule
\end{tabularx}

\subsection{План реагирования на риски}

\noindent
\begin{tabularx}{\textwidth}{lXX}
\toprule
\textbf{Риск} & \textbf{План действий} & \textbf{Участники и роли} \\
\midrule
\textbf{R1} & Разработка системы кэширования данных на стороне сервера и локальной БД. Использование заглушек (mock-data) при сбоях. & \textbf{Архитектор (Ковальчук А.И.)}: проектирование кэша. \textbf{Разработчик (Смирнов Н.Д.)}: реализация фолбэк-логики. \\
\midrule
\textbf{R2} & Проведение R\&D исследования на раннем этапе. Упрощение логики синхронизации в пользу базового кэширования. & \textbf{Тимлид (Коновалов И.А.)}: исследование Android API, разработка модуля синхронизации. \\
\midrule
\textbf{R3} & Взаимозаменяемость участников (cross-functional). Хранение всей документации и кода в актуальном состоянии в GitHub. & \textbf{Продакт-менеджер (Воронин Г.Д.)}: контроль актуальности документации. \textbf{Тимлид (Коновалов И.А.)}: перераспределение задач. \\
\midrule
\textbf{R4} & Регулярный сбор обратной связи от целевой группы (студенты) на этапе разработки MVP. Корректировка UX. & \textbf{Продакт-менеджер (Воронин Г.Д.)}: организация интервью. \textbf{Дизайнер (Николаев Р.А.)}: оперативная правка интерфейса. \\
\midrule
\textbf{R5} & Проведение еженедельных код-ревью и встреч по синхронизации интерфейсов API. & \textbf{Архитектор (Ковальчук А.И.)}: аудит кода. \textbf{Разработчик (Смирнов Н.Д.)}, \textbf{Дизайнер (Николаев Р.А.)}: согласование контрактов. \\
\bottomrule
\end{tabularx}

\section{Откорректированный план разработки проекта}

План скорректирован с учетом заложенных временных буферов на минимизацию рисков R1 и R2. В таблице приведено сравнение исходной и новой длительности этапов.

\noindent
\begin{tabularx}{\textwidth}{p{3.5cm}ccp{2.5cm}X}
\toprule
\textbf{Этап} & \textbf{Старая длит.} & \textbf{Новая длит.} & \textbf{Даты} & \textbf{Обоснование} \\
\midrule
\textbf{1. Подготовительный} & 5 дней & 5 дней & 01--05.12.2025 & \textbf{Без изменений.} Базовое проектирование. \\
\midrule
\textbf{2. Backend разработка} & 18 дней & 22 дня & 06--27.12.2025 & \textbf{Сдвиг внутренних задач (R1).} Добавлены задачи по кэшированию и отказоустойчивости API. \\
\midrule
\textbf{3. Frontend разработка} & 27 дней & 27 дней & 08.12--15.01.2026 & \textbf{Без изменений.} Разработка идет параллельно Backend. \\
\midrule
\textbf{4. Основные функции} & 30 дней & 30 дней & 28.12--27.01.2026 & \textbf{Без изменений.} Плановый объем работ. \\
\midrule
\textbf{5. Android разработка} & 25 дней & 30 дней & 10.01--13.02.2026 & \textbf{Увеличение длительности (R2).} Добавлено время на R\&D и отладку офлайн-режима. \\
\midrule
\textbf{6. Тестирование} & 15 дней & 20 дней & 25.01--14.02.2026 & \textbf{Сдвиг начала (R1, R2, R5).} Расширен объем регрессионного тестирования из-за сложности синхронизации. \\
\midrule
\textbf{7. Завершение} & 5 дней & 5 дней & 15--19.02.2026 & \textbf{Без изменений.} Финальная документация и деплой. \\
\bottomrule
\end{tabularx}

\section{Бюджет разработки}

\noindent
\begin{tabularx}{\textwidth}{Xp{7cm}r}
\toprule
\textbf{Статья расходов} & \textbf{Описание} & \textbf{Сумма (руб.)} \\
\midrule
\textbf{ФОТ: Тимлид / Android-dev} & 2.5 месяца разработки (Коновалов И.А.) & 625 000 \\
\textbf{ФОТ: Архитектор / Backend} & 2.5 месяца разработки (Ковальчук А.И.) & 575 000 \\
\textbf{ФОТ: Продакт-менеджер} & 2.5 месяца (Воронин Г.Д.) & 450 000 \\
\textbf{ФОТ: Backend-разработчик} & 2.5 месяца (Смирнов Н.Д.) & 500 000 \\
\textbf{ФОТ: Frontend / Дизайнер} & 2.5 месяца (Николаев Р.А.) & 450 000 \\
\midrule
\textbf{Инфраструктура} & Хостинг (Vercel, MongoDB Atlas), домены, SSL & 50 000 \\
\textbf{Лицензии и API} & Подписки на ПО, платные лимиты Google Books API & 45 000 \\
\textbf{Маркетинг и запуск} & Первичное продвижение среди студентов & 150 000 \\
\textbf{Резервный фонд} & Рисковый буфер (15\% от бюджета) & 426 750 \\
\midrule
\textbf{ИТОГО} & & \textbf{3 271 750} \\
\bottomrule
\end{tabularx}

\section{Обоснование статей расходов}

\begin{enumerate}
    \item \textbf{Фонд оплаты труда (ФОТ):} Составляет основную часть бюджета, так как проект требует высокой квалификации специалистов для реализации сложной логики аннотирования и кроссплатформенной синхронизации. Ставки соответствуют рыночным для middle-специалистов.
    \item \textbf{Инфраструктура и API:} Выбранные сервисы (Vercel, MongoDB) обеспечивают масштабируемость и отказоустойчивость системы. Затраты на API необходимы для легального и стабильного доступа к книжному контенту.
    \item \textbf{Маркетинг:} Сумма заложена на привлечение первых 1000 пользователей, что является KPI проекта для подтверждения ценности (Product-Market Fit).
    \item \textbf{Резервный фонд:} Необходим для покрытия расходов в случае реализации выявленных рисков (R1, R2, R3), требующих дополнительных человеко-часов.
\end{enumerate}

\section{План оптимизации расходов}

\begin{enumerate}
    \item \textbf{Использование Free-tier:} На этапе разработки и закрытого бета-тестирования использовать бесплатные тарифные планы облачных сервисов, что сэкономит до 40 000 руб.
    \item \textbf{Поэтапный найм:} Привлечение дополнительных специалистов только на пиковые периоды (например, интенсивная фаза Android-разработки).
    \item \textbf{Open Source решения:} Максимальное использование бесплатных библиотек для модуля чтения, что исключает необходимость покупки дорогих коммерческих SDK.
    \item \textbf{Фокус на MVP:} Отказ от второстепенных функций (социальные ленты, расширенная аналитика) до подтверждения гипотез проекта, что сокращает трудозатраты на 20\%.
\end{enumerate}

\end{document}
