\documentclass[12pt,a4paper]{article}
\usepackage[utf8]{inputenc}
\usepackage[russian]{babel}
\usepackage{geometry}
\usepackage{fancyhdr}
\usepackage{enumitem}
\usepackage{booktabs}
\usepackage{xcolor}
\usepackage{longtable}
\usepackage{array}
\usepackage{graphicx}
\usepackage{float}

\geometry{margin=2.5cm}
\pagestyle{fancy}
\fancyhf{}
\fancyhead[L]{Исследование пользовательских сценариев}
\fancyhead[R]{Проект «BookHub»}
\fancyfoot[C]{\thepage}

\title{\textbf{Исследование пользовательских сценариев}}
\author{}
\date{}

\begin{document}

\maketitle

\vspace{1cm}

\section{Определение ключевых пользовательских сценариев}

На основе анализа функциональных требований MVP были выделены следующие ключевые пользовательские сценарии:

\begin{enumerate}
    \item \textbf{Регистрация нового пользователя}
    \item \textbf{Авторизация существующего пользователя}
    \item \textbf{Поиск книги в каталоге}
    \item \textbf{Открытие книги для чтения}
    \item \textbf{Создание аннотации к фрагменту текста}
    \item \textbf{Просмотр своих аннотаций}
    \item \textbf{Создание новой коллекции}
    \item \textbf{Добавление книги в коллекцию}
    \item \textbf{Просмотр публичных аннотаций других пользователей}
\end{enumerate}

\vspace{1cm}

\section{Customer Journey Map}

\subsection{Таблица сценариев}

\begin{figure}[H]
\centering
\includegraphics[width=\textwidth]{CJM.png}
\caption{Таблица сценариев Customer Journey Map}
\end{figure}

\vspace{1cm}

\section{Анализ количества шагов по сценариям}

\subsection{Статистика по сценариям}

\begin{table}[h]
\centering
\begin{tabular}{lcc}
\toprule
\textbf{Сценарий} & \textbf{Количество шагов} & \textbf{Категория сложности} \\
\midrule
Регистрация & 8 & Средняя \\
Авторизация & 4 & Простая \\
Поиск книги & 7 & Средняя \\
Открыть книгу & 6 & Средняя \\
Создать аннотацию & 12 & Сложная \\
Просмотр своих аннотаций & 7 & Средняя \\
Создать коллекцию & 7 & Средняя \\
Добавить в коллекцию & 8 & Средняя \\
Просмотр публичных аннотаций & 8 & Средняя \\
\bottomrule
\end{tabular}
\end{table}

\subsection{Расчет среднего числа шагов}

\textbf{Среднее арифметическое:} 7.4 шага

\textbf{Медиана:} 7 шагов

\subsection{Выводы по анализу}

\begin{itemize}
    \item \textbf{Самый простой сценарий:} Авторизация (4 шага) — базовый, часто используемый.
    \item \textbf{Самый сложный сценарий:} Создание аннотации (12 шагов) — ключевая функция продукта.
    \item \textbf{Общая оценка:} Среднее число шагов 7.4 — приемлемо для MVP, но есть потенциал для оптимизации, особенно длинных сценариев.
\end{itemize}

\vspace{1cm}

\section{Петли и тупиковые ситуации}

\subsection{Обнаруженные проблемы}

\subsubsection{Онбординг-петля}

\textbf{Проблема:} Новый пользователь должен зарегистрироваться, чтобы увидеть ценность продукта. Если нет возможности просмотреть каталог без регистрации, пользователь может уйти.

\textbf{Решение:} Реализовать гостевой режим с ограниченным доступом к каталогу и чтению (например, первые 3 страницы книги).

\subsubsection{Навигационная петля в создании аннотации}

\textbf{Проблема:} При создании аннотации пользователь должен:

\begin{enumerate}
    \item Найти книгу в каталоге
    \item Открыть книгу
    \item Найти нужный фрагмент
    \item Выделить текст
    \item Создать заметку
\end{enumerate}

Если пользователь не помнит, на какой странице нужный фрагмент, он может застрять в цикле поиска.

\textbf{Решение:}

\begin{itemize}
    \item Добавить закладки для быстрого перехода к важным местам
    \item Реализовать поиск по тексту внутри книги
    \item Сохранять последнюю позицию чтения
\end{itemize}

\vspace{1cm}

\section{Варианты оптимизации пользовательских сценариев}

\subsection{Оптимизация сценария "Создание аннотации"}

\textbf{Текущий процесс (12 шагов):}

\begin{enumerate}
    \item Зайти на сайт
    \item Авторизоваться
    \item Перейти в каталог
    \item Найти нужную книгу
    \item Нажать на книгу
    \item Нажать "Читать"
    \item Найти нужный фрагмент
    \item Выделить фрагмент текста
    \item Нажать "Создать заметку"
    \item Ввести текст заметки
    \item Выбрать уровень доступа
    \item Нажать "Сохранить"
\end{enumerate}

\textbf{Оптимизированный процесс (8 шагов):}

\begin{enumerate}
    \item Зайти на сайт
    \item Авторизоваться
    \item Открыть книгу из истории чтения / закладок / коллекции
    \item Выделить фрагмент текста
    \item Ввести текст заметки (появляется inline-редактор)
    \item Выбрать уровень доступа (по умолчанию "Приватный")
    \item Нажать "Сохранить" (или Enter)
    \item Продолжить чтение
\end{enumerate}

\textbf{Оптимизации:}

\begin{itemize}
    \item \textbf{Быстрый доступ к книгам:} История чтения, закладки, избранное на главной странице
    \item \textbf{Inline-редактор:} Появляется сразу при выделении текста, без дополнительных кликов
    \item \textbf{Умные значения по умолчанию:} Приватный доступ выбран автоматически
    \item \textbf{Сохранение позиции:} Автоматическое сохранение последней позиции чтения
\end{itemize}

\textbf{Экономия:} 4 шага

\vspace{1cm}

\subsection{Оптимизация сценария "Добавление книги в коллекцию"}

\textbf{Текущий процесс (8 шагов):}

\begin{enumerate}
    \item Зайти на сайт
    \item Авторизоваться
    \item Перейти в каталог
    \item Найти нужную книгу
    \item Нажать на книгу
    \item Нажать "Добавить в коллекцию"
    \item Выбрать коллекцию из списка
    \item Подтвердить добавление
\end{enumerate}

\textbf{Оптимизированный процесс (5 шагов):}

\begin{enumerate}
    \item Зайти на сайт
    \item Авторизоваться
    \item Найти книгу (через поиск или в каталоге)
    \item Нажать кнопку "Добавить в коллекцию" (на карточке книги)
    \item Выбрать коллекцию (выпадающий список без подтверждения)
\end{enumerate}

\textbf{Оптимизации:}

\begin{itemize}
    \item \textbf{Кнопка на карточке:} Добавление доступно прямо из списка книг, без открытия детальной страницы
    \item \textbf{Без подтверждения:} Добавление происходит сразу после выбора коллекции
    \item \textbf{Быстрое создание:} Возможность создать новую коллекцию прямо из выпадающего списка
\end{itemize}

\textbf{Экономия:} 3 шага

\vspace{1cm}

\subsection{Оптимизация сценария "Поиск книги"}

\textbf{Текущий процесс (7 шагов):}

\begin{enumerate}
    \item Зайти на сайт
    \item Авторизоваться
    \item Перейти в каталог
    \item Ввести запрос в поиск
    \item Нажать "Поиск"
    \item Выбрать книгу из результатов
    \item Нажать на книгу
\end{enumerate}

\textbf{Оптимизированный процесс (5 шагов):}

\begin{enumerate}
    \item Зайти на сайт
    \item Авторизоваться
    \item Ввести запрос в поиск (на главной странице или в шапке)
    \item Выбрать книгу из результатов (автодополнение / выпадающий список)
    \item Нажать на книгу (или Enter для перехода к первой книге)
\end{enumerate}

\textbf{Оптимизации:}

\begin{itemize}
    \item \textbf{Поиск на главной:} Поисковая строка доступна на главной странице, не нужно переходить в каталог
    \item \textbf{Автодополнение:} Результаты появляются при вводе, без нажатия кнопки "Поиск"
    \item \textbf{Быстрый переход:} Enter или клик сразу открывает книгу
\end{itemize}

\textbf{Экономия:} 2 шага

\vspace{1cm}

\subsection{Общие оптимизации для всех сценариев}

\subsubsection{Сохранение состояния сессии}

\begin{itemize}
    \item \textbf{Проблема:} Пользователь должен авторизовываться при каждом входе
    \item \textbf{Решение:} Сохранение сессии, автоматическая авторизация
    \item \textbf{Экономия:} 1-2 шага для повторных визитов
\end{itemize}

\subsubsection{Умная главная страница}

\begin{itemize}
    \item \textbf{Проблема:} Главная страница не предоставляет быстрого доступа к часто используемым функциям
    \item \textbf{Решение:}
    \begin{itemize}
        \item История чтения (последние 5 книг)
        \item Рекомендации на основе активности
        \item Быстрый поиск
        \item Последние аннотации
    \end{itemize}
    \item \textbf{Экономия:} 2-3 шага для частых действий
\end{itemize}

\end{document}

