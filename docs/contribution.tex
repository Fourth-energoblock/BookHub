\documentclass[12pt,a4paper]{article}
\usepackage[utf8]{inputenc}
\usepackage[russian]{babel}
\usepackage{geometry}
\usepackage{fancyhdr}
\usepackage{enumitem}
\usepackage{booktabs}
\usepackage{xcolor}
\usepackage{listings}

\geometry{margin=2.5cm}
\pagestyle{fancy}
\fancyhf{}
\fancyhead[L]{Правила работы с репозиторием}
\fancyhead[R]{Проект «BookHub»}
\fancyfoot[C]{\thepage}

\title{\textbf{Правила работы с репозиторием проекта "BookHub"}}
\author{}
\date{}

\begin{document}

\maketitle

\vspace{1cm}

Этот документ описывает основные правила и процессы, принятые в команде "Fourth energoblock" для работы с данным репозиторием. Соблюдение этих правил обязательно для всех участников проекта.

\section{Общий процесс внесения изменений}

Любые изменения — будь то добавление документации, исправление ошибок или любая другая задача — должны вноситься через \textbf{Pull Request (PR)}. Прямые коммиты в ветку \texttt{main} \textbf{запрещены} настройками репозитория.

\textbf{Порядок действий для каждого участника:}

\begin{enumerate}
    \item \textbf{Синхронизация:} Перед началом любой работы убедитесь, что ваша локальная ветка \texttt{main} содержит самые последние изменения с GitHub.
    \begin{lstlisting}[language=bash]
    git checkout main
    git pull origin main
    \end{lstlisting}

    \item \textbf{Создание новой ветки:} Для каждой новой задачи создавайте отдельную ветку от \texttt{main}. Название ветки должно соответствовать правилам именования (см. раздел 2).
    \begin{lstlisting}[language=bash]
    git checkout -b <nazvanie-vashej-vetki>
    \end{lstlisting}

    \item \textbf{Работа над задачей:} Внесите необходимые изменения в файлы проекта.

    \item \textbf{Коммиты:} Сделайте один или несколько коммитов, чтобы сохранить ваши изменения. Сообщения коммитов должны соответствовать принятому стилю (см. раздел 3).
    \begin{lstlisting}[language=bash]
    git add .
    git commit -m "tip: kratkoe opisanie vashih izmenenij"
    \end{lstlisting}

    \item \textbf{Отправка на GitHub:} Отправьте вашу ветку с коммитами в удаленный репозиторий.
    \begin{lstlisting}[language=bash]
    git push origin <nazvanie-vashej-vetki>
    \end{lstlisting}

    \item \textbf{Создание Pull Request:} На сайте GitHub появится предложение создать Pull Request из вашей ветки. Нажмите на него.
    \begin{itemize}
        \item \textbf{Целевая ветка:} \texttt{main}.
        \item \textbf{Название PR:} Кратко и понятно опишите суть всех изменений.
        \item \textbf{Reviewers:} В правой панели назначьте всех остальных участников команды для проверки вашего кода.
    \end{itemize}
\end{enumerate}

\section{Правила именования веток}

Название ветки должно давать понять, над какой задачей ведется работа. Мы используем следующие префиксы:

\begin{itemize}
    \item \texttt{feature/} — для создания новых артефактов или добавления крупных блоков информации (например, \texttt{feature/create-questionnaire}).
    \item \texttt{docs/} — для обновления документации (например, \texttt{docs/update-contribution-guide}).
    \item \texttt{fix/} — для исправления ошибок или опечаток (например, \texttt{fix/correct-typos-in-readme}).
\end{itemize}

\textbf{Пример правильного названия:} \texttt{feature/add-communication-plan}

\section{Стиль именования коммитов}

Мы придерживаемся стандарта \textbf{Conventional Commits}. Это помогает нам поддерживать историю изменений чистой и понятной, а также в будущем позволит автоматизировать создание отчетов об изменениях.

\textbf{Формат сообщения:} \texttt{<тип>: <краткое описание>}

\textbf{Основные типы:}
\begin{itemize}
    \item \textbf{feat}: Добавление нового файла-артефакта или раздела в документ.
    \item \textbf{docs}: Изменения, касающиеся только документации (например, обновление \texttt{README.md} или этого файла).
    \item \textbf{fix}: Исправление опечаток, ошибок в форматировании.
    \item \textbf{style}: Изменения, не влияющие на содержание (пробелы, форматирование и т.д.).
\end{itemize}

\textbf{Примеры сообщений:}
\begin{itemize}
    \item \texttt{feat: Add initial draft of meeting protocol}
    \item \texttt{docs: Describe branch naming rules}
    \item \texttt{fix: Correct contact info in communication plan}
\end{itemize}

\section{Система одобрений (Approve'ов)}

Для обеспечения высокого качества нашей работы и общего понимания процесса, каждое изменение должно быть проверено командой.

\begin{itemize}
    \item \textbf{Правило:} Для слияния Pull Request в ветку \texttt{main} необходимо получить \textbf{одобрение (Approve)} от всех остальных участников команды.
    \item \textbf{Расчет:} Так как в нашей команде 5 человек, для слияния PR необходимо получить \textbf{4 одобрения}.
    \item \textbf{Ограничение:} Автор Pull Request'а не может одобрить собственный PR.
\end{itemize}

\textbf{Процесс ревью:}
\begin{enumerate}
    \item После создания PR автор запрашивает ревью у всех членов команды.
    \item Ревьюеры просматривают изменения, оставляют комментарии, если есть вопросы или замечания.
    \item Если изменения соответствуют требованиям, ревьюер ставит "Approve".
    \item После получения 4-х одобрений PR может быть успешно слит в \texttt{main}.
\end{enumerate}

\end{document}
