\documentclass[12pt,a4paper]{article}
\usepackage[utf8]{inputenc}
\usepackage[russian]{babel}
\usepackage{geometry}
\usepackage{fancyhdr}
\usepackage{enumitem}
\usepackage{booktabs}
\usepackage{xcolor}

\geometry{margin=2.5cm}
\pagestyle{fancy}
\fancyhf{}
\fancyhead[L]{Анкета для сбора первичных требований}
\fancyhead[R]{Проект «BookHub»}
\fancyfoot[C]{\thepage}

\title{\textbf{Анкета для сбора первичных требований к проекту "BookHub"}}
\author{}
\date{18.10.2025}

\begin{document}

\maketitle

\vspace{1cm}

\begin{center}
\textbf{Название проекта:} BookHub – Платформа для социального чтения \\
\textbf{Заказчик:} Команда "Квартет П" \\
\textbf{Исполнитель:} Команда "4 энергоблок" \\
\textbf{Дата:} 18.10.2025 \\
\textbf{Версия документа:} 1.0
\end{center}

\vspace{1cm}

\section*{Преамбула}

Уважаемая команда "Квартет П",

Цель данной анкеты — сбор и формализация требований для разработки минимально жизнеспособного продукта (MVP) и последующего развития платформы "BookHub". Ваши точные и развернутые ответы позволят нам определить ключевые параметры проекта и сформировать четкое техническое задание.

\vspace{1cm}

\section*{Раздел 1: Стратегическое видение и цели проекта}

\begin{enumerate}
    \item Какое ключевое конкурентное преимущество BookHub вы выделяете по сравнению с платформами Bookmate и Goodreads? Какую основную потребность целевой аудитории продукт удовлетворяет наиболее эффективно?
    
    \item Опишите целевой пользовательский опыт (User Experience) при первом взаимодействии с платформой. Какую основную ценность продукта пользователь должен осознать в первые минуты использования?
    
    \item Какие ключевые показатели эффективности (KPI) будут определять успешность MVP по истечении первых 6 месяцев после запуска?
    
    \item Уточните, в каком виде будет представлен конечный результат "коллективного знания". Предполагается ли создание общей базы аннотаций, версионируемого документа или иного цифрового артефакта?
\end{enumerate}

\vspace{1cm}

\section*{Раздел 2: Функциональные требования}

\subsection*{2.1. Основной функционал (Контент и библиотека)}

\begin{enumerate}[resume]
    \item Какие форматы цифровых книг (например, EPUB, PDF, FB2) должны поддерживаться платформой на этапе MVP?
    
    \item Предусматривает ли архитектура продукта возможность загрузки пользователями собственного контента, или доступ будет ограничен библиотекой платформы?
    
    \item Предполагается ли на первом этапе интеграция с конкретными источниками контента или API открытых цифровых библиотек? Если да, то с какими?
\end{enumerate}

\subsection*{2.2. Социальное взаимодействие}

\begin{enumerate}[resume]
    \item Помимо функционала цитирования и рецензирования, какие дополнительные механизмы социального взаимодействия между пользователями необходимо реализовать?
    
    \item Опишите модель функционирования "групп и клубов для чтения". Какими правами и инструментами модерации должны обладать их создатели?
    
    \item На каких наборах данных должна основываться система персональных рекомендаций (например, история чтения, оценки, социальный граф, иные параметры)?
\end{enumerate}

\subsection*{2.3. Совместная работа с текстом}

\begin{enumerate}[resume]
    \item Опишите предполагаемую модель данных для совместного аннотирования. Будут ли аннотации существовать в виде единого для всех документа или в виде индивидуальных слоев с настраиваемой видимостью?
    
    \item Требуется ли синхронизация аннотаций и комментариев в реальном времени (real-time) для всех участников сессии?
    
    \item Необходимо ли реализовать систему контроля версий для документов с совместными аннотациями, позволяющую просматривать историю изменений и восстанавливать предыдущие состояния?
\end{enumerate}

\vspace{1cm}

\section*{Раздел 3: Целевая аудитория и монетизация}

\begin{enumerate}[resume]
    \item Какая из четырех перечисленных целевых аудиторий является приоритетным сегментом для MVP?
    
    \item Какой специализированный функционал требуется для сегмента "студенты и преподаватели", отсутствующий у других групп пользователей?
    
    \item Определите, пожалуйста, модель ограничений для бесплатной версии продукта (freemium). Будет ли это ограничение по функционалу, объему контента, времени использования или иная модель?
    
    \item Определен ли целевой ценовой диапазон или модель ценообразования для Premium-подписки?
    
    \item Какой специализированный инструментарий необходим для корпоративных (B2B) клиентов в рамках их командных пространств?
\end{enumerate}

\vspace{1cm}

\section*{Раздел 4: Технические требования и UX/UI}

\begin{enumerate}[resume]
    \item Существуют ли у вас строгие предпочтения или технические ограничения по выбору технологического стека из представленных в презентации пар (React/Vue.js, Node.js/Python)?
    
    \item Подразумевает ли "мобильная версия" на этапе MVP создание адаптивного веб-приложения (responsive web design) или разработку нативных мобильных приложений для iOS/Android?
    
    \item Какие аспекты безопасности (например, защита персональных данных, управление цифровыми правами (DRM) на контент) являются наиболее критичными для проекта?
    
    \item Существуют ли у вас корпоративные гайдлайны (брендбук), определяющие визуальный стиль проекта, или иные требования к UI/UX дизайну?
\end{enumerate}

\vspace{1cm}

\section*{Раздел 5: План развития и управление проектом}

\begin{enumerate}[resume]
    \item Перечислите, пожалуйста, 3-5 ключевых функций, которые являются абсолютно обязательными для запуска MVP. Какой функционал должен быть реализован в первую очередь?
    
    \item Какой предполагается регламент сбора и обработки обратной связи на этапах альфа- и бета-тестирования?
    
    \item Оцените степень гибкости сроков, указанных в плане развития. Какие факторы или риски могут оказать на них наибольшее влияние?
\end{enumerate}

\vspace{2cm}

\section*{Подписи сторон}

\textbf{Подпись со стороны Заказчика ("Квартет П"):}

\vspace{2cm}

\rule{5cm}{0.4pt} / Рабинович М. /

\vspace{1cm}

\textbf{Подпись со стороны Исполнителя ("4 энергоблок"):}

\vspace{2cm}

\rule{5cm}{0.4pt} / Коновалов И.А. /

\end{document}
