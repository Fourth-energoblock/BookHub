\documentclass[12pt,a4paper]{article}
\usepackage[utf8]{inputenc}
\usepackage[russian]{babel}
\usepackage{geometry}
\usepackage{fancyhdr}
\usepackage{enumitem}
\usepackage{booktabs}
\usepackage{xcolor}
\usepackage{array}

\geometry{margin=2.5cm}
\pagestyle{fancy}
\fancyhf{}
\fancyhead[L]{План коммуникаций}
\fancyhead[R]{Проект «BookHub»}
\fancyfoot[C]{\thepage}

\title{\textbf{План коммуникаций по проекту "BookHub"}}
\author{}
\date{}

\begin{document}

\maketitle

\vspace{1cm}

\begin{center}
\textbf{Название проекта:} BookHub – Платформа для социального чтения \\
\textbf{Документ:} План регламента коммуникаций \\
\textbf{Дата:} [Укажите текущую дату] \\
\textbf{Версия:} 1.0
\end{center}

\vspace{1cm}

\section{Цель документа}

Настоящий документ определяет порядок и правила взаимодействия между командой Заказчика ("Квартет П") и командой Исполнителя ("4 энергоблок") на всех этапах разработки проекта "BookHub". Целью является обеспечение прозрачности, своевременного обмена информацией и эффективного решения рабочих вопросов.

\section{Участники проекта и контактные лица}

\subsection{Команда Исполнителя ("4 энергоблок")}

\begin{table}[h]
\centering
\begin{tabular}{|p{3cm}|p{4cm}|p{1.5cm}|p{3cm}|}
\hline
\textbf{Роль} & \textbf{ФИО} & \textbf{Группа} & \textbf{Контакты (Email/Telegram)} \\
\hline
Тимлид & Коновалов Иван Андреевич & 243 & @knvlvivn \\
\hline
Продакт-менеджер & Воронин Глеб Дмитриевич & 243 & @glebb98 \\
\hline
Архитектор & Ковальчук Артем Игоревич & 243 & @kandler3 \\
\hline
Разработчик & Смирнов Никита Денисович & 243 & @abkubyb \\
\hline
Дизайнер & Николаев Роман Андреевич & 243 & @niklvrr \\
\hline
\end{tabular}
\end{table}

\subsection{Команда Заказчика ("Квартет П")}

\begin{table}[h]
\centering
\begin{tabular}{|p{3cm}|p{4cm}|p{1.5cm}|}
\hline
\textbf{Роль} & \textbf{ФИО} & \textbf{Группа} \\
\hline
Руководитель проекта & Рабинович Майя & 243 \\
\hline
Технический писатель & Двойцов Егор Андреевич & 248 \\
\hline
Дизайнер & Германов Артём Дмитриевич & 248 \\
\hline
Архитектор / Аналитик & Лещук Глеб Олегович & 249 \\
\hline
\end{tabular}
\end{table}

\section{Каналы и регламент коммуникаций}

\subsection{Еженедельные статус-встречи}
\begin{itemize}
    \item \textbf{Цель:} Обсуждение прогресса за неделю, планирование задач на следующую неделю, решение возникших проблем.
    \item \textbf{Формат:} Онлайн-встреча в Google Meet.
    \item \textbf{Время:} Каждый понедельник, 11:00 (МСК).
    \item \textbf{Обязательные участники:} Тимлид и Продакт-менеджер со стороны Исполнителя, Представители Заказчика.
\end{itemize}

\subsection{Оперативное общение}
\begin{itemize}
    \item \textbf{Цель:} Решение срочных вопросов, быстрые согласования, неформальное общение.
    \item \textbf{Канал:} Общий чат в Telegram.
\end{itemize}

\subsection{Управление задачами и документацией}
\begin{itemize}
    \item \textbf{Цель:} Постановка и отслеживание выполнения задач, хранение артефактов проекта.
    \item \textbf{Канал:} Репозиторий проекта на GitHub. Все задачи и их статусы отслеживаются через Pull Requests и Issues. Вся документация (анкета, протоколы) хранится в папках \texttt{docs/} и \texttt{content/}.
\end{itemize}

\subsection{Экстренные ситуации}
\begin{itemize}
    \item \textbf{Цель:} Решение критических проблем, блокирующих работу.
    \item \textbf{Канал:} Прямой звонок или личное сообщение Тимлиду (Исполнитель) или Представителю Заказчика.
\end{itemize}

\section{Порядок согласования и утверждения}

Все изменения и артефакты проекта (анкеты, протоколы, документация) утверждаются через механизм Pull Request в GitHub. Изменение считается принятым после получения необходимого количества одобрений (Approvals) от уполномоченных лиц со стороны обеих команд.

\section{Подписи сторон}

Настоящий план вступает в силу с момента его подписания уполномоченными представителями обеих сторон.

\textbf{Со стороны Заказчика ("Квартет П"):}

\vspace{2cm}

\rule{5cm}{0.4pt} / Рабинович Майя /

\vspace{1cm}

\textbf{Со стороны Исполнителя ("4 энергоблок"):}

\vspace{2cm}

\rule{5cm}{0.4pt} / Коновалов Иван Андреевич /

\end{document}
