\documentclass[12pt,a4paper]{article}
\usepackage[utf8]{inputenc}
\usepackage[russian]{babel}
\usepackage{geometry}
\usepackage{fancyhdr}
\usepackage{enumitem}
\usepackage{booktabs}
\usepackage{xcolor}

\geometry{margin=2.5cm}
\pagestyle{fancy}
\fancyhf{}
\fancyhead[L]{Протокол встречи по результатам интервью}
\fancyhead[R]{Проект «BookHub»}
\fancyfoot[C]{\thepage}

\title{\textbf{Протокол встречи по результатам интервью для проекта «BookHub»}}
\author{}
\date{20.10.2025}

\begin{document}

\maketitle

\vspace{1cm}

\begin{center}
\textbf{Дата:} 20.10.2025 \\
\textbf{Формат:} Онлайн (видеоконференция) \\
\textbf{Проект:} BookHub
\end{center}

\vspace{1cm}

\section*{Присутствовали:}

\subsection*{Со стороны Исполнителя (Команда «4 энергоблок»):}
\begin{itemize}[leftmargin=2cm]
    \item Коновалов Иван Андреевич (Тимлид)
    \item Воронин Глеб Дмитриевич (Продакт-менеджер)
    \item Ковальчук Артем Игоревич (Архитектор)
    \item Смирнов Никита Денисович (Разработчик)
    \item Николаев Роман Андреевич (Дизайнер)
\end{itemize}

\subsection*{Со стороны Заказчика (Команда «Квартет П»):}
\begin{itemize}[leftmargin=2cm]
    \item Рабинович Майя (Руководитель проекта)
    \item Двойцов Егор Андреевич (Технический писатель)
    \item Германов Артём Дмитриевич (Дизайнер)
    \item Лещук Глеб Олегович (Архитектор / Аналитик)
\end{itemize}

\vspace{1cm}

\section*{Повестка дня:}

\begin{enumerate}
    \item Обсуждение вопросов из анкеты для уточнения требований к проекту.
    \item Фиксация ключевых требований к MVP (Minimum Viable Product).
    \item Определение границ и функционала проекта на первом этапе (0-6 месяцев).
\end{enumerate}

\vspace{1cm}

\section*{Ход обсуждения и принятые решения:}

По результатам обсуждения были зафиксированы следующие ключевые требования и решения, которые станут основой для разработки Технического Задания (ТЗ) на MVP.

\subsection*{1. Стратегия и фокус MVP (0–6 месяцев)}

\begin{itemize}
    \item \textbf{Приоритетная аудитория:} Активная молодежь (студенты и начинающие специалисты), читающие цифровые книги и ищущие социальное взаимодействие вокруг текста.
    \item \textbf{Ключевые метрики успеха (KPI):}
    \begin{itemize}
        \item 1000 регистраций за первые 3 месяца;
        \item Retention Rate > 40\%;
        \item Среднее время в приложении $\approx$ 15 минут;
        \item Не менее 500 созданных заметок/аннотаций;
        \item Доступность системы $\geq$ 99.9\%.
    \end{itemize}
\end{itemize}

\subsection*{2. Функциональность и контент}

\begin{itemize}
    \item \textbf{Базовый функционал MVP:}
    \begin{itemize}
        \item Регистрация и авторизация;
        \item Поиск по книгам;
        \item Чтение текстовых книг;
        \item Создание аннотаций и заметок;
        \item Организация книг в коллекции;
        \item Публичные, приватные и доступные по ссылке заметки;
        \item Веб-версия и Android-клиент с офлайн-доступом.
    \end{itemize}
    \item \textbf{Функции, не входящие в MVP:} Загрузка собственных книг и совместное аннотирование в реальном времени.
    \item \textbf{Интеграция с библиотеками:} Project Gutenberg, Google Books API (для метаданных), LitRes (публичные коллекции).
\end{itemize}

\subsection*{3. Ключевые технические решения}

\begin{itemize}
    \item \textbf{Технологический стек:} Frontend: React; Backend: Node.js (Express) + MongoDB; Деплой: Vercel / Cloud Functions.
    \item \textbf{Реализация аннотирования:} В MVP — только индивидуальные заметки с тремя уровнями доступа.
\end{itemize}

\subsection*{4. Монетизация и привлечение пользователей}

\begin{itemize}
    \item \textbf{Ограничения бесплатной версии:} Доступ к классике, до 10 коллекций и 100 приватных заметок.
    \item \textbf{Стратегия привлечения:} Кампании в соцсетях, партнерства с университетами и блогерами.
\end{itemize}

\subsection*{5. Конкурентное преимущество}

\begin{itemize}
    \item \textbf{Отличие от Glose:} Фокус на взаимодействии \textbf{внутри} текста и интеграция с RU-библиотеками.
    \item \textbf{Отличие от Goodreads:} Соцсеть внутри книги, а не об отзывах, с геймификацией и локализованным контентом.
\end{itemize}

\subsection*{6. Юридические вопросы и безопасность}

\begin{itemize}
    \item \textbf{Авторские права:} Ответственность за контент несет пользователь. Требуется подготовка соответствующего Пользовательского соглашения.
\end{itemize}

\vspace{2cm}

\section*{Подписи сторон}

Стороны подтверждают, что данный протокол точно отражает все достигнутые договоренности и принятые решения.

\vspace{1cm}

\textbf{От Исполнителя (Команда «4 энергоблок»):}

\vspace{2cm}

\rule{5cm}{0.4pt} / Коновалов И.А. (Тимлид) /

\vspace{1cm}

\textbf{От Заказчика (Команда «Квартет П»):}

\vspace{2cm}

\rule{5cm}{0.4pt} / Рабинович М. (Руководитель проекта) /

\end{document}
