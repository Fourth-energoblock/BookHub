\documentclass[12pt,a4paper]{article}
\usepackage[utf8]{inputenc}
\usepackage[russian]{babel}
\usepackage{geometry}
\usepackage{fancyhdr}
\usepackage{enumitem}
\usepackage{booktabs}
\usepackage{xcolor}

\geometry{margin=2.5cm}
\pagestyle{fancy}
\fancyhf{}
\fancyhead[L]{Техническое Задание}
\fancyhead[R]{Проект «BookHub»}
\fancyfoot[C]{\thepage}

\title{\textbf{Техническое Задание (ТЗ) на разработку MVP проекта «BookHub»}}
\author{}
\date{21.10.2025}

\begin{document}

\maketitle

\vspace{1cm}

\begin{center}
\textbf{Статус:} Утверждено \\
\textbf{Версия:} 1.0 \\
\textbf{Дата:} 21.10.2025
\end{center}

\vspace{1cm}

\section{Общие сведения}

\subsection{Название проекта}
BookHub

\subsection{Основание для разработки}
Протокол встречи от 20.10.2025, зафиксировавший требования Заказчика.

\subsection{Цель проекта}
Разработать MVP (Minimum Viable Product) онлайн-сервиса, объединяющего функциональность цифровой библиотеки и социальной сети для совместного чтения.

\subsection{Приоритетная аудитория MVP}
Активная молодежь (студенты и начинающие специалисты), использующая цифровые книги.

\subsection{Границы MVP}
Данное ТЗ описывает исключительно функционал, входящий в первую версию продукта (MVP). Все функции, не перечисленные в данном документе, считаются выходящими за рамки MVP.

\section{Бизнес-цели и метрики успеха (KPI)}

Продукт будет считаться успешным при достижении следующих показателей в течение 3 месяцев после запуска:

\begin{itemize}
    \item \textbf{Регистрации:} не менее 1000 пользователей.
    \item \textbf{Удержание (Retention Rate):} $> 40\%$.
    \item \textbf{Вовлеченность (Engagement):} Среднее время сессии в приложении $\approx$ 15 минут.
    \item \textbf{Активность:} Создано не менее 500 публичных или приватных аннотаций.
\end{itemize}

\section{Функциональные требования}

\subsection{Пользовательская система (Регистрация и Авторизация)}
\begin{itemize}
    \item Пользователь может зарегистрироваться с помощью Email и пароля.
    \item Пользователь может войти в свою учетную запись.
    \item Пользователь может выйти из своей учетной записи.
    \item Авторизация должна быть реализована с использованием JWT (JSON Web Token).
\end{itemize}

\subsection{Каталог книг}
\begin{itemize}
    \item Пользователь может видеть список доступных книг.
    \item Пользователь может осуществлять поиск книг по названию, автору.
    \item Для каждой книги в каталоге отображается обложка, название, автор.
\end{itemize}

\subsection{Модуль чтения (Ридер)}
\begin{itemize}
    \item Пользователь может открыть книгу для чтения.
    \item Пользователь может листать страницы книги.
    \item \textbf{Для Android-клиента:} должна быть обеспечена возможность офлайн-доступа к нескольким книгам, добавленным пользователем.
\end{itemize}

\subsection{Система аннотаций и заметок}
\begin{itemize}
    \item Во время чтения пользователь может выделить фрагмент текста.
    \item К выделенному фрагменту пользователь может создать текстовую заметку (аннотацию).
    \item Пользователь может установить для своей заметки один из трех уровней доступа:
    \begin{enumerate}
        \item \textbf{Приватный:} видна только автору.
        \item \textbf{Публичный:} видна всем пользователям платформы.
        \item \textbf{По ссылке:} видна только тем, у кого есть специальная ссылка на заметку.
    \end{enumerate}
    \item Пользователь может просматривать, редактировать и удалять свои заметки.
\end{itemize}

\subsection{Личные коллекции}
\begin{itemize}
    \item Пользователь может создавать персональные коллекции (например, "Прочитать позже", "Любимые").
    \item Пользователь может добавлять книги из каталога в свои коллекции.
    \item Пользователь может удалять книги из своих коллекций.
\end{itemize}

\section{Нефункциональные требования}

\begin{itemize}
    \item \textbf{Доступность:} Доступность системы должна быть не менее 99.9\%.
    \item \textbf{Безопасность:}
    \begin{itemize}
        \item Взаимодействие клиента и сервера должно осуществляться по протоколу HTTPS.
        \item Пароли пользователей должны храниться в зашифрованном виде (хэш).
    \end{itemize}
    \item \textbf{Поддерживаемые платформы:}
    \begin{itemize}
        \item Веб-приложение (последние версии Chrome, Firefox, Safari).
        \item Мобильное приложение для Android (версия 8.0 и выше).
    \end{itemize}
\end{itemize}

\section{Требования к контенту}

Интеграция для получения контента и метаданных должна быть реализована со следующими источниками:

\begin{itemize}
    \item \textbf{Project Gutenberg} (для книг в общественном достоянии).
    \item \textbf{Google Books API} (для получения метаданных: обложки, аннотации, рейтинг).
    \item \textbf{LitRes} (для доступа к публичным и бесплатным коллекциям).
\end{itemize}

\section{Что не входит в MVP}

Следующий функционал \textbf{не является} частью MVP и запланирован на будущие версии:

\begin{itemize}
    \item Загрузка пользователями собственных книг в любых форматах (fb2, ePub, pdf).
    \item Совместное редактирование аннотаций в реальном времени.
    \item Аудиокниги.
    \item Система личных сообщений между пользователями.
    \item Продвинутая система рекомендаций на основе машинного обучения.
\end{itemize}

\section{Согласование}

Данное Техническое Задание является официальным руководством для разработки MVP проекта «BookHub».

\textbf{От Исполнителя (Команда «4 энергоблок»):}

\vspace{2cm}

\rule{5cm}{0.4pt} / Коновалов И.А. (Тимлид) /

\vspace{1cm}

\textbf{От Заказчика (Команда «Квартет П»):}

\vspace{2cm}

\rule{5cm}{0.4pt} / Рабинович М. (Руководитель проекта) /

\end{document}
